\documentclass[12pt, a4paper]{scrartcl}

\newcommand{\datetimenow}{2022.02.16}

\sloppy

\renewcommand{\familydefault}{\ttdefault}
\newcommand{\defaultfamily}{\ttfamily}
\renewcommand{\bfseries}{\fontseries{bx}\selectfont}
\DeclareRobustCommand{\bxseries}{\fontseries{bx}\selectfont}
\DeclareRobustCommand{\textbx}{\fontseries{bx}}

\catcode`_=12 %
\renewcommand{\texttt}[1]{%
  \begingroup
  \ttfamily
  \begingroup\lccode`~=`/\lowercase{\endgroup\def~}{/\discretionary{}{}{}}%
  \begingroup\lccode`~=`[\lowercase{\endgroup\def~}{[\discretionary{}{}{}}%
  \begingroup\lccode`~=`.\lowercase{\endgroup\def~}{.\discretionary{}{}{}}%
  \begingroup\lccode`~=`_\lowercase{\endgroup\def~}{_\discretionary{}{}{}}%
  \catcode`/=\active\catcode`[=\active\catcode`.=\active\catcode`_=\active
  \scantokens{#1\noexpand}%
  \endgroup
}
\catcode`_=8 %

\usepackage[utf8]{inputenc}
\usepackage[english]{babel}
%\usepackage{ascii}
\usepackage[OT1]{fontenc}
\usepackage{tgbonum}
\usepackage{amsmath}
\usepackage{amssymb}
\usepackage{amstext}
\usepackage{amsfonts}
\usepackage{array}
\usepackage{mathrsfs}
%\usepackage{physics}
\usepackage[version=4]{mhchem}
\usepackage{chemfig}
\usepackage{ulem}
\usepackage{romannum}
\usepackage{graphicx}
\usepackage{siunitx}
\usepackage{xparse}
\usepackage{tabularx}
\usepackage{makecell}
\usepackage{multirow}
\usepackage{wrapfig}
\sisetup{
	range-phrase = {\ \linebreak[0]\text{to}\ \nolinebreak},
	%scientific-notation = engineering,
	%locale = EN,
	per-mode = fraction,
	separate-uncertainty = true
}
\DeclareSIUnit\clight{{\mathrm{c}_{\text{0}}}}
\DeclareSIUnit\atomicmassunit{\mathrm{u}}
\usepackage{framed}
\usepackage{xcolor}
\usepackage{pgfplots, pgfplotstable}
\usepackage{booktabs}
%\usepackage{parskip}
\usepackage{float}
\usepackage{microtype}
\usepackage{subfigure}
\bibliographystyle{unsrt}
\usepackage{imakeidx}
\usepackage{url}
\usepackage{tabulary}
\usepackage[
backend=biber,
style = nature,
sorting=ynt
]{biblatex}
\addbibresource{Physic_formulary.bib}
\setcounter{biburllcpenalty}{7000}
\setcounter{biburlucpenalty}{8000}
\renewbibmacro*{title}{%
	\ifboolexpr{
		test {\iffieldundef{title}}
		and
		test {\iffieldundef{subtitle}}
	}
	{}
	{\printtext{%
		\printtext[titlecase]{\usefield{\uline}{title}}%
		\setunit{\subtitlepunct}%
		\printfield[titlecase]{subtitle}}%
		\newunit%
	}%
	\printfield{titleaddon}%
}
%\usepackage{overcite}
\usepackage{hyperref}
\usepackage{blindtext}
\usepackage{tikz}
\setlength{\headheight}{1,5cm}
\usepackage{geometry}
\geometry{hmargin=2cm, vmargin=3.5cm}
%\usepackage{fancyhdr}
%\pagestyle{fancy}
%\usepackage{titlesec}
\usepackage{scrlayer-scrpage}
\usepackage{bold-extra}
%\titleformat{\section}{\defaultfamily\bxseries\Large}{\thesection}{1em}{}
%\titleformat{\subsection}{\defaultfamily\bxseries\large}{\thesubsection}{1em}{}
%\titleformat{\subsubsection}{\defaultfamily\bxseries}{\thesubsubsection}{1em}{}
%\titleformat{\paragraph}{\defaultfamily\bxseries}{\theparagraph}{1em}{}

\addtokomafont{section}{\defaultfamily\bxseries\Large}
\addtokomafont{subsection}{\defaultfamily\bxseries\large}
\addtokomafont{subsubsection}{\defaultfamily\bxseries}
\addtokomafont{paragraph}{\defaultfamily\bxseries}
%\renewcommand*{\thesection}{\arabic{section}}

%\setlength\extrarowheight{8pt}
\renewcommand{\arraystretch}{1.25}
\newenvironment{formulatable}{%
	\begingroup%
	\setlength{\tabcolsep}{0pt}%
	\tabularx{\textwidth}{%
		>{\hsize = 1.5\hsize\raggedleft\arraybackslash\begin{math}} X <{\:\end{math}}%
		>{\hsize = 0.75\hsize\raggedright\arraybackslash\begin{math}} X <{\end{math}}%
		>{\hsize = 0.75\hsize\raggedright\arraybackslash\:}X%
	}%
}{%
	\endtabularx%
	\endgroup%
}
\newcommand{\formulatableheading}[1]{\multicolumn{3}{c}{#1}\\}

\makeindex[intoc = true, options = {-s Physic_formulary.ist}]

\title{\Large{\textbf{Physic formulary}}}
\author{School}
\date{\datetimenow}

\begin{document}
\begin{titlepage}
\pagenumbering{Roman}

\maketitle
\thispagestyle{empty}
\end{titlepage}

\newpage
\ihead{School}
\chead{Physic formulary}
\ohead{\datetimenow}

\tableofcontents

\newpage

\pagenumbering{arabic}
\section[Constants]{Constants\cite{NIST-Constants}}
\begin{tabularx}{\textwidth}{l l @{{} $ \: = \: $ {}}X}
	Bohr radius\index{Bohr radius}			&	$a_{\text{0}}$		& \begin{math} \frac{4\pi \varepsilon_{\text{0}}\hbar^{2}}{m_{\text{e}}e^{2}} = \frac{\varepsilon_{\text{0}}h^{2}}{\pi m_{\text{e}}e^{2}} = \frac {\hbar }{m_{\text{e}}c\alpha} \* = \qty{5.29177210903e-11}{\metre} \end{math}\\
	Velocity of light\index{Light}:			&	$\unit{\clight}$	& $\qty{299792458}		{\meter\per\second}$\\
	Elementary charge:				&	$e$			& $\qty{1.602176634e-19}	{\coulomb}$\\
	Vacuum permittivity:				&	$\varepsilon_\text{0}$
											& $\qty{8.8541878128e-12}	{\farad\per\metre}$\\
	Permittivity of air:				&	$\varepsilon_\text{r}$
											& $\qty{1.00059}		{}$\\
	Faraday constant{\index{Faraday constant}}	&	$F$			& $\qty{96485.3321233}		{\coulomb\per\mole}$\\
	Acceleration due to gravity:			&	$g$			& $\qty{9.80665}		{\metre\per\square\second}$\\
	Gravitational constant\index{Gravitation!Gravitational constant}:
							&	$G$			& $\qty{6.67430e-11}		{\cubic\meter\per\kilogram\per\square\second}$\\
	Planck constant\index{Plank constant}:		&	$h$			& $\qty{6.62607015e-34}		{\joule\per\hertz}$\\
	Boltzmann constant\index{Boltzmann constant}:	&	$k$			& $\qty{1.280649e-23}		{\joule\per\kelvin}$\\
	\multirow{2}{*}{Electron\index{Electron} mass:}	&	$m_\text{e}$		& $\qty{9.1093837015e-31}	{\kilogram}$\\
							&	$m_\text{e}$		& $\qty{0.51099895000}		{\mega\electronvolt\per\square\clight}$\\
	\multirow{3}{*}{Muon\index{Elemtary particle!Leptons!Muon} mass:}
							&	$m_{\text{$\mu$}}$	& $\qty{1.883531627e-28}	{\kilogram}$\\
							&	$m_{\text{$\mu$}}$	& $\qty{105.6583755}		{\mega\electronvolt\per\square\clight}$\\
							&	$m_{\text{$\mu$}}$	& $\qty{0.1134289259}		{\dalton}$\\
	\multirow{2}{*}{Neutron mass:}			&	$m_\text{n}$		& $\qty{1.67492749804e-27}	{\kilogram}$\\
							&	$m_\text{n}$		& $\qty{939.56542052}		{\mega\electronvolt\per\square\clight}$\\
	\multirow{2}{*}{Proton mass:}			&	$m_\text{p}$		& $\qty{1.67262192369e-27}	{\kilogram}$\\
							&	$m_\text{p}$		& $\qty{938.27208816}		{\mega\electronvolt\per\square\clight}$\\
	Vacuum permeability:				&	$\mu_\text{0}$		& $\qty{1.25663706212e-6}	{\henry\per\metre}$\\
	Permeability of air:				&	$\mu_\text{r}$		& $\qty{1.00000037}		{}$\\
\end{tabularx}

\newpage
\section{Other physical interrelationships}
\begin{tabularx}{\textwidth}{l l >{\raggedright\arraybackslash} X}
	Visible spectrum:				& \multicolumn{2}{l}{$ \qtyrange{380}{750}{\nano\metre}$}\\
	Speed of sound under standart conditions:	& \multicolumn{2}{l}{$ \qty{343}{\metre\per\second}$}\\
	Dalton $\si{\dalton}$ / unified atomic mass unit $\unit{\atomicmassunit}$:
							&	$\unit{\dalton}/\unit{\atomicmassunit}$
											& \begin{math}\sloppy= \qty{1.66053906660e-27}	{\kilo\gram}\end{math}\\
	Hydrogen mass:					&	$m_\text{H}$		& \begin{math}\sloppy= \qtyrange{1.00784}{1.00811}{\dalton}\end{math}\\
	Atomic mass of helium $\ce{^{4}He}$		&	$m_\text{He}$		& $= \qty{4.002603254}		{\dalton}$\\
	Kilowatt-hour:					&	$\mathrm{kW \, h}$	& $= \qty{3.6e6}		{\joule}$\\
	Kilowatt-hour:					&	$\unit{\electronvolt}$	& $= \qty{1.602176634e-19}{\joule}$\\
	Pressure:					&	$\qty{1}{\pascal}$	& $= \qty{1}{\newton\per\square\metre}$\\
	Pressure:					&	$\qty{1}{bar}$		& $= \qty{e5}{\pascal}$\\
	Absolute zero:					&	$\qty{-273.15}{\degreeCelsius}$
											& $= \qty{0}{\kelvin}$\\
\end{tabularx}

\newpage
\section{Energy\index{Energy}}
\begin{formulatable}
	\toprule
	\formulatableheading{Kinetic Energy\index{Energy!Kinetic Energy}}
	\midrule
	E_{\text{k}}	&	= \frac{1}{2} m v^{2} & ~\\
	\formulatableheading{}
	\formulatableheading{Potential Energy\index{Energy!Potential Energy}}
	\midrule
	U		&	= m g h		& (gravitational)\\
	U 		&	= \frac{1}{2} \cdot k \cdot x^{2}
						& (elastic)\\
	U		&	= \frac{1}{2} \cdot C \cdot V^{2}
						& (electric)\\
	U		&	= -m B		& (magnetic)\\
	U		&	= \int F(r) \, dr	& (general)\\
	\bottomrule
\end{formulatable}
\section{Motion\index{motion}}
\begin{formulatable}
	\toprule
	\multicolumn{3}{c}{uniform linear motion\index{motion!uniform linear motion}}\\
	\midrule
	s(t)		& = v t \, \left( \, + s_{\text{0}} \, \right)	& ~\\
	v(t)		& = const.					& ~\\
	a(t)		& = 0\\
	\multicolumn{3}{c}{}\\
	\multicolumn{3}{c}{non-uniform linear motion\index{motion!non-uniform linear motion}}\\
	\midrule
	s(t)		& = \frac{1}{2} a t^{2} \left(\, + v_{\text{0}} \, t + s_{\text{0}} \, \right)
									& ~\\
	v(t)		& = a t \, \left( \, + v_{\text{0}} \, \right)	& ~\\
	a(t)		& = const.					& ~\\
	\multicolumn{3}{c}{}\\
	\multicolumn{3}{c}{circular motion\index{motion!circular motion}}\\
	\midrule
	F_{\text{Z}}	& = \frac{m v^{2}}{r} = m \, \omega^{2} r		& ~\\
	\omega		& = \frac{v}{r} = \frac{2\pi}{T} = \frac{\Delta\varphi}{\Delta t}
									& $\varphi$ in rad\\
	f		& = \frac{1}{T}					& ~\\
	\bottomrule
\end{formulatable}

\section{Momentum\index{Momentum}}
\begin{formulatable}
	\toprule
	\multicolumn{3}{c}{momentum itself}\\
	\midrule
	\vec{p}		& = m \vec{v} & ~\\
	\vec{p}		& = {\frac{\hbar}{\lambda}}			& photons\\
	\vec{p}		& = \sqrt{m_{\text{0}}^{2} \unit{\square\clight} + \frac{E^{2}}{\unit{\square\clight}}}
									& general\\
	\multicolumn{3}{c}{}\\
	\midrule
	\multicolumn{3}{c}{relations to momentum}\\
	\midrule
	\sum_{i} m_{i} \, u_{i}
			& = \sum_{i} m_{i} \, v_{i}		& conservation of momentum\\
	\Delta p	& = F \Delta t & ~\\
	E_{\text{k}}	& = \int p \, dv & ~\\
\end{formulatable}
\section{Electricity\index{Electromagnetism}}
\begin{formulatable}
	\toprule
	\multicolumn{3}{c}{General}\\
	\midrule
	I		& = \frac{\Delta Q}{\Delta t} = \frac{\partial Q}{\partial t} = \dot Q & in \unit{\ampere}\\
	R		& = \frac{U}{I} = \rho \frac{l}{A}		& in \unit{\ohm}\\
	E		& = U \cdot Q					&~\\
	P		& = \frac{\Delta E}{\Delta t} = U I		& in \unit{\watt} Energy flow / "Power"\\
\end{formulatable}

\newpage
\section{Fields}
\subsection{Newtonian gravitation\index{Gravitation}}
\begin{formulatable}
	\toprule
	\multicolumn{3}{c}{Homogeneous field\index{Gravitation!Homogeneous Field}}\\
	\midrule
	E		& = mgh						& ~\\
	\vec{F_{g}}	& = m \vec{g}					& ~\\
	\vec{g}		& = \frac{\vec{F_{g}}}{m}				& ~\\
	\Delta \varphi	& = \frac{E}{m} = gh				& ~\\
	\multicolumn{3}{c}{}\\
	\midrule
	\multicolumn{3}{c}{Radial symmetric field\index{Gravitation!Radial symmetric field}}\\
	\midrule
	U		& = G Mm \left( \frac{1}{r_{1}} - \frac{1}{r_{2}} \right)
									& ~\\
	U		& = - G Mm \frac{1}{r}				& for $r_{1} \to \infty$\\
	\vec{F_{g}}	& = - G \frac{Mm}{r^{2}} \hat{r}			& ~\\
	\vec{g}		& = \frac{\vec{F_{g}}}{m} = -G \frac{M}{r_{2}} \hat r\\
	\Delta \varphi	& = \frac{E}{m} = \gamma M \left( \frac{1}{r_{1}} - \frac{1}{r_{2}} \right)
									& ~\\
\end{formulatable}
\subsection{Electromagnetism\index{Electromagnetism}\index{Electromagnetism!Electrostatics}}
\begin{formulatable}
	\toprule
	\formulatableheading{homogenous field\index{Electromagnetism!Electrostatics!Homogenous field}}
	\midrule
	E		& = q \vec{\mathbf E} d				& ~\\
	\vec{F}		& = q \vec{\mathbf E}				& ~\\
	\vec{\mathbf E}	& = \frac{\vec{F}}{q}				& ~\\
	\mathbf E	& = \frac{V}{d}					& ~\\
	V		& = \frac{E}{q} = \vec{\mathbf E} d		& ~\\
	\formulatableheading{}
	\midrule
	\formulatableheading{Radial symetric field\index{Elecromagnetism!Electrostatics!Radial symetric field}}
	\midrule
	U		& = - \frac{1}{4 \pi \varepsilon_{0} \varepsilon_{r}} Qq \left( \frac{1}{r_{1}} - \frac{1}{r_{2}} \right)
									& ~\\
	U		& = \frac{Qq}{4 \pi \varepsilon_{0}} \frac{1}{r}	& for $r_{1} \to \infty$\\
	\vec{F}		& = \frac{1}{4 \pi \varepsilon_{0} \varepsilon_{r}} \frac{Qq}{r^{2}} \hat r
									& ~\\
	\mathbf E	& = \frac{\vec{F}}{q} = \frac{1}{4 \pi \varepsilon_{0} \varepsilon{r}} \frac{Q}{r^{2}} \hat r
									& ~\\
	\varepsilon_{0}	& = \frac{\sigma}{\varepsilon_{r} \mathbf E}	& const.\\
\end{formulatable}

\newpage
\section[Units]{Units\index{Units}\cite{NIST-Units}}
\begingroup
\begin{tabularx}{\textwidth}{>{\raggedright\arraybackslash}X | l | >{\begin{math}} r <{\end{math}} @{\hskip 0pt}@{{}$\:${}} >{\begin{math}} l <{\end{math}}| >{\begin{math}} r <{\end{math}}}
	Derived quantity	& Name		& \multicolumn{2}{l}{In terms of other SI units\:} \vline & \mathtt{Dimensions}\\
	\toprule
	electric charge, quantity of electricity
				& coulomb\index{Units!coulomb}	& \unit{\coulomb}	& = \unit{\second\ampere}								& {\mathsf{T\;I}}\\
	capacitance		& farad\index{Units!farad}	& \unit{\farad}		& = \unit{\second\tothe{4}\per\square\metre\per\kilogram\square\ampere}			& {\mathsf{T^{4}\;L^{-2}\;M^{-1}\;I^{2}}}\\
	inductance		& henry\index{Units!henry}	& \unit{\henry}		& = \unit{\per\square\second\square\metre\kilogram\per\square\ampere}			& {\mathsf{T^{-2}\;L^{2}\;M\;I^{-1}}}\\
	frequency		& hertz\index{Units!hertz}	& \unit{\hertz}		& = \unit{\per\second}									& {\mathsf{T^{-1}}}\\
	energy, work, quantity of heat
				& joule\index{Units!joule}	& \unit{\joule}		& = \unit{\per\square\second\square\metre\kilogram}					& {\mathsf{T^{-2}\;L^{2}\;M}}\\
	force			& newton\index{Units!newton}	& \unit{\newton}	& = \unit{\per\square\second\metre\kilogram}						& {\mathsf{T^{-2}\;L\;M}}\\
	electric resistance	& ohm\index{Units!ohm}		& \unit{\ohm}		& = \unit{\per\cubic\second\square\metre\kilogram\square\ampere}			& {\mathsf{T^{-3}\;L^{2}\;M\;I^{-2}}}\\
	pressure, stress	& pascal\index{Units!pascal}	& \unit{\pascal}	& = \unit{\newton\per\metre} = \unit{\per\square\second\per\metre\kilogram}		& {\mathsf{T^{-2}\;L^{-1}\;M}}\\
	plane angle		& radian\index{Units!radian}	& \unit{\radian}	& = \unit{\metre\per\metre}								& {\mathsf{}}\\
	magnetic flux		& weber\index{Units!weber}	& \unit{\tesla}		& = \unit{\weber\per\square\metre} = \unit{\per\square\second\kilogram\per\ampere}	& {\mathsf{T^{-2}\;M\;I^{-1}}}\\
	electric potential difference, electromotive force
				& volt\index{Units!volt}	& \unit{\volt}		& = \unit{\watt\per\ampere} = \unit{\kilogram\square\metre\per\cubic\second\per\ampere}	& {\mathsf {T^{-3}\;L^{2}\;M\;I^{-1}}}\\
	power, radiant flux	& watt\index{Units!watt}	& \unit{\watt}		& = \unit{\joule\per\second} = \unit{\per\cubic\second\square\metre\kilogram}		& {\mathsf{T^{-3}\;L^{2}\;M}}\\
\end{tabularx}
\endgroup

\printindex
\newpage
{\raggedright\printbibliography[heading=bibintoc, title={Bibliography}]}

\end{document}
